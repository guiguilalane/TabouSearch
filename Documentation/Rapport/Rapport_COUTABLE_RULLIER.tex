%% Classe du document
\documentclass[a4paper,10pt]{article}

%% Francisation
\usepackage[francais]{babel} % Indique que l'on utilise le francais
\usepackage[T1]{fontenc}
\usepackage[utf8]{inputenc} % Indique que l'on utilise tout le clavier
%\usepackage[latin1]{inputenc}

%% Réglages généraux
\usepackage[top=3cm, bottom=3cm, left=3cm, right=3cm]{geometry} % Taille de la feuille
\usepackage{lastpage}

%% Package pour le texte
\usepackage{soul} % Souligner
\usepackage{color} % Utilisation de couleurs
\usepackage{hyperref} % Créer des liens et des signets
\usepackage{eurosym}% Pour le symbole euro
\usepackage{fancyhdr}% Entête et pied de page

%% Package pour les tableaux
\usepackage{multirow} % Colonnes multiples
\usepackage{cellspace}
\usepackage{array}

%% Package pour les dessins
\usepackage{pstricks}
\usepackage{graphicx} % Importer des images
\usepackage{pdftricks} % Pour utiliser avec pdfTex
\usepackage{pst-pdf} % Pour utiliser avec pdfTex
\usepackage{pst-node} % Pose de noeuds
\usepackage{subfig}
\usepackage{float}

%% Package pour les maths
\usepackage{amsmath} % Commandes essentielles
\usepackage{amssymb} % Principaux symboles

%% Package pour le code
\usepackage{listings} % Utilisation de la couleur syntaxique des langages
\usepackage{url}


\usepackage[babel=true]{csquotes} % Permet les quotations (guillemets)
\usepackage{tocvsec2}
\usepackage{amsthm}
\usepackage{amsfonts}

\usepackage{tikz}
\usepackage{pdfpages}

\usetikzlibrary{shapes} % A revoir

%--------------------- Autres définitions ---------------------%

% Propriété des liens
\hypersetup{
colorlinks = true, % Colorise les liens
urlcolor = blue, % Couleur des hyperliens
linkcolor = black, % Couleur des liens internes
}

\definecolor{grey}{rgb}{0.95,0.95,0.95}

% Language Definitions for Turtle
%TODO: a revoir avec les couleur de gedit
\definecolor{olivegreen}{rgb}{0.2,0.8,0.5}
\definecolor{grey2}{rgb}{0.5,0.5,0.5}
\lstdefinelanguage{ttl}{
sensitive=true,
morecomment=[s][\color{grey2}]{@}{:},
morecomment=[l][\color{olivegreen}]{\#},
morecomment=[s][\color{red}]{<}{/>},
morestring=[s][\color{olivegreen}]{<http://w}{\#>},
morestring=[b][\color{blue}]{\"},
}

\lstset{%language = sql,
basicstyle =\footnotesize,
%numbers = left,
numberstyle=\normalsize,
numbersep=10pt,
framexleftmargin=5mm,
frame=lines,
tabsize=4}

%Definition de la commande pour retirer l'espace devant les ':'
\makeatletter
\@ifpackageloaded{babel}%
        {\newcommand{\nospace}[1]{{\NoAutoSpaceBeforeFDP{}#1}}}%  % !! double {{}} pour cantonner l'effet à l'argument #1 !!
        {\newcommand{\nospace}[1]{#1}}
\makeatother

\setcounter{tocdepth}{3}
%\maxsecnumdepth{subsubsection} % Dernière section numérotée

\newcommand{\paperPrototyping}{\emph{paper prototyping}}

% Corps du document :
\begin{document}

% Définition des entêtes et pieds de page
\fancyhead[LE,CE,RE,LO,CO,RO]{}
\fancyfoot[LE,CE,RE,LO,CO,RO]{}
\fancyhead[LO, LE]{Programmation par contraintes}
\fancyhead[RO,RE]{2012/2013}
\fancyfoot[LO,LE]{Université de \scshape{Nantes}}
\fancyfoot[RO,RE]{Page \thepage \ sur \pageref{LastPage}}
\renewcommand{\headrulewidth}{0.4pt}
\renewcommand{\footrulewidth}{0.4pt}

%\maketitle
\begin{titlepage}

\vspace*{\fill}~
\begin{center}
{\large \textsc{Rapport de Projet}} \\
\textsc{Tabu Search} \\
\vspace{0.5cm}
COUTABLE Guillaume, RULLIER Noémie \\
\today
\end{center}
\vspace*{\fill}

\vspace{\stretch{1}}
\begin{center}
\noindent 
\includegraphics[height=2.5cm]{Images/universite.png}
\end{center}
\pagebreak
\end{titlepage}

\newpage
\tableofcontents  

% Introduction
\newpage
\pagestyle{fancy}

%%%%%%%%%%%%%%%%%%%%%%%%%%%%%%%%%%%%%%%%%%%%%%%%%%%%%%%%%%%%%%%%%%%%%%%%%%%%%
%%%%%%%%%%  Introduction générale
%%%%%%%%%%%%%%%%%%%%%%%%%%%%%%%%%%%%%%%%%%%%%%%%%%%%%%%%%%%%%%%%%%%%%%%%%%%%%
\section{Introduction}
L'objectif de ce TP fut de comprendre et d'implémenter l'algorithme TabuSearch en JaCoP.

%%%%%%%%%%%%%%%%%%%%%%%%%%%%%%%%%%%%%%%%%%%%%%%%%%%%%%%%%%%%%%%%%%%%%%%%%%%%%
%%%%%%%%%% TS
%%%%%%%%%%%%%%%%%%%%%%%%%%%%%%%%%%%%%%%%%%%%%%%%%%%%%%%%%%%%%%%%%%%%%%%%%%%%%
\section{Présentation du TabuSearch}
%TODO
Tabu search est une heuristique de la méthode de recherche locale. Elle consiste dans un premier temps à générer affectation totale des variables. Elle va ensuite effectuer le mouvement qui a un coût le moins important. Tabu search permet d'ajouter une contrainte, en effet on marque les $p$ derniers mouvements effectués qui seront interdits. On répète cette dernière opération jusqu'à que le coût soit égal à 0 ou jusqu'à ce que le nombre maximum d'essai soit atteint.
%TODO --> Voir pour les aspirations

\section{N-Queens avec TabuSearch et CompleteSearch comparaison}

\section{Ajout des conditions d'aspirations}

\section{TabuSearch algorithme avec procédure de redémarrage}

\end{document}

\begin{figure}[H]
    \center
    \includegraphics[width=2cm]{Images/menuFichier.png}
    \caption{Le menu Fichier}
\end{figure}